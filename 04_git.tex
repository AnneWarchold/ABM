% The Slide Definitions
\input{templates/slide_definitions}

% Author and Course information
\input{templates/slide_information}

\newtoggle{Diplom} \togglefalse{Diplom}
\newtoggle{Bachelor} \togglefalse{Bachelor}
\toggletrue{\abschluss}

\newcommand{\git}[0]{\texttt{git} }

% Presentation title
\title{Git}
\date{\today}


\begin{document}

\maketitle

\begin{frame}{Gliederung}
	\setbeamertemplate{section in toc}[sections numbered]
	\tableofcontents
\end{frame}


\section{Mathe-Prüfung}
\begin{frame}{Nikolausklausur}
    ... die erste Mathe-Klausur naht!
    \begin{itemize}
        \item \textbf{Termin:} \texttt{10.12.2020 16:40 Uhr}
	\item \textbf{Einschreibung:} endet am \texttt{7.12.2020}
        \item Hausaufgaben bringen Bonuspunkte!
        \item zeitig (am besten \emph{vorgestern}) mit Lernen anfangen
        \item lernt zusammen!
	\item \textbf{Hilfsmittel:} ein A4-Blatt ein- oder beidseitig handschriftlich beschrieben
    \end{itemize}

    \vfill

    \centering \LARGE \alert{\textbf{Eine 2.0 jetzt sichert euch das Modul!}}
    % TODO: Größe so lassen?
\end{frame}

%\section{Prüfungs- und Studienordnung}
%\input{\abschluss-Ordnung}

%\section{Allgemeine Prüfungstipps}

%\begin{frame}{Lerntyp finden}
%	Habt ihr bestimmt schon in der Schule gemacht - wenn nicht:
%	\begin{itemize}
%		\item eigenes Lernverhalten reflektieren: Was klappt bei mir gut?
%		\item Google liefert zahlreiche Infos und Selbsttests
%	\end{itemize}
%	Die vier klassischen Lerntypen:
%	\begin{description}[kommunikativ]
%		\item [auditiv] Lernen durch Hören und Erzählen%, ruhige Umgebung ist wichtig
%		\item [visuell] Lernen durch Schreiben und mit Hilfe von Übersichten% und Bildern, ordentliche Umgebung ist wichtig
%		\item [motorisch] Lernen durch Anfassen (z.B. Modelle) und Selbermachen %Bewegung beim Lernen (an der Tafel, Prüfungssituation nachstellen)
%		\item [kommunikativ] Lernen durch Interaktion und Austausch mit Anderen %Diskutieren und Fragen Stellen
%	\end{description}
%	\alert{Zu wissen, wie euer Hirn \glqq{}tickt\grqq{}, macht euch das Lernen leichter!}
%\end{frame}

%\begin{frame}{Hilfe! In meiner WG ist immer Party, wo soll ich lernen?!}
%	\vspace{.5cm}
%	Mögliche Orte zum Lernen und Arbeiten (gemeinsam oder allein):
%	\begin{itemize}
%		\item \alert{Seminarräume} (vor allem am WE/in der vorlesungsfreien Zeit)\\
%		$\longrightarrow$ ggf. vom Wachpersonal aufschließen lassen
%		\item freie \alert{PC-Pools} (Öffnungszeiten des ZIH beachten)
%		\item verstreute Tische in den \alert{oberen Etagen des APB}
%		\item in der \alert{SLUB} (Gruppenarbeitsräume, ruhige Ecken oder belebterer Eingangsbereich)
%	\end{itemize}
%		\vspace{1cm}
%		\cleanchapterquote{Und wenn du da ein kleines bisschen flüsterst, wirst du von Leuten mit Blicken angestarrt, die dich TÖTEN!!! ... Ist geil!}{{\small Anonymer FSRling}}{{\small über den Zentralen Lesesaal der SLUB}}
%\end{frame}

%\begin{frame}{Und wann sollte ich anfangen mit Lernen?}
%	\centering \LARGE \alert{Am ersten Tag der Vorlesungszeit!}\\
%	\Large Das Studium ist ein Vollzeitjob:\\[.7cm]
%	\flushleft\normalsize
%	\begin{description}
%		\item [1 LP] = 30 h Arbeitszeit
%		\item [1 Semester] = 30 LP und somit 900 h Arbeitszeit
%		\item [Vorlesungs- und Kernprüfungszeit] ~\\19 Wochen (ohne Weihnachten) im WiSe 18/19, \\17 Wochen (ohne Pfingsten) im SoSe 19
%	\end{description}
%	\vspace{.5cm}
%	$\Longrightarrow$ \alert{47,4 h} bzw. sogar \alert{52,9 h} Arbeitszeit pro Woche!
%\end{frame}

%\begin{frame}
%	\centering \includegraphics[width=.95\textwidth, keepaspectratio]{assets/04_git/what}
%	
%	\Huge What?!
%\end{frame}

%\begin{frame}
%	\centering\Large\alert{Zum Glück braucht man für die meisten Module weniger Zeit}\\[.3cm]
%	\flushleft\normalsize
%	Dennoch: Unterschätzt den Aufwand nicht, und fangt nicht erst zwei Wochen vor der Klausur mit Lernen an!\\[.7cm]
%	\alert{Tipps:}
%	\begin{itemize}
%		\item Altbekannt, aber dennoch zu selten umgesetzt: Lernpläne, Zeitpläne und ToDo-Listen schreiben
%		\item TimeTracker (z.B. aTimeLogger für Android) nutzen und Ziel setzen, wie viele Stunden pro Woche man mit Uni verbringen will
%	\end{itemize}
%\end{frame}

%\begin{frame}{Hilfsmittel in Prüfungen}
%	Im Bezug auf erlaubte Hilfsmittel ist jede Prüfung anders.
%	\begin{description}
%		\item [Mathe] (beide Prüfungen)
%		\begin{itemize}
%			\item ein A4-Blatt, beidseitig \textbf{hand}beschrieben
%			\item \underline{kein} Taschenrechner
%		\end{itemize}
%		\item [AuD] gar nichts
%		\item [EMI] nicht-programmierbarer Taschenrechner
%		\item [TGI] ein A4-Blatt, beidseitig bedruckt
%		\item [RA] je Vorlesung ein beidseitig bedrucktes A4-Blatt,\\aber Prüfung immer erst im Sommersemester % TODO: Wird das dieses Semester angeboten? (ist für diese Folie aber erstmal egal, die Hilfsmittel-Info ist dennoch sinnvoll)
%	\end{description}
	
%	\centering\alert{Alle Angaben ohne Gewähr! ;)}
%	\note{
%		\begin{itemize}
%			\item Darauf hinweisen, dass sich sowas auch immer mal wieder ändert, z.B. Mathe: mal alles, mal nur ein Blatt
%		\end{itemize}}
%\end{frame}

%\begin{frame}{Korrektoren sind auch nur Menschen}
%	\centering \LARGE Geht zu den \alert{Prüfungseinsichten}! \\[.7cm]
%
%	\normalsize Die Korrektoren sind auch nur Menschen. Wer in wenigen Tagen mehrere hundert Klausuren korrigiert, kann auch mal einen Fehler machen oder etwas übersehen.
%\end{frame}


\section{Git}
\subsection{Warum git?}

\begin{frame}
	\centering \Huge \alert{Was ist git?}\\[.7cm]
	\normalsize
	\begin{itemize}
		\item Ein Versionsverwaltungssystem!
	\end{itemize}
	\centering \Huge \alert{Und wozu ist das gut?}\\[.7cm]
	\normalsize
	\begin{itemize}
		\item Um den Überblick über verschiedene Versionen deines Projektes zu behalten
		\item Zum einfacheren Zusammenarbeiten mehrerer Personen an einem gemeinsamen Projekt
		\begin{itemize}
			\item Deshalb im Robolab und SWP genutzt
		\end{itemize}
	\end{itemize}
\end{frame}

\begin{frame}
	\centering \Huge \textbf{Warum sollte ich git nutzen?} \\[.7cm]

	\normalsize Was spricht gegen Dropbox? Oder Facebook?
\end{frame}

\begin{frame}
	\centering \includegraphics[width=.9\textwidth, keepaspectratio]{assets/04_git/codesalat}

	\uncover<2->{\Large \textbf{\alert{Das.}}}
\end{frame}

\begin{frame}{Warum Versionierung?}
	\begin{itemize}
		\item Dropbox ist (für solche Zwecke) unbenutzbar
		\begin{itemize}
			\item man überschreibt gegenseitig Änderungen oder hat 20 Dateien à la \\
				\centering \texttt{(In Konflikt stehende Version von xx am dd.mm.yyyy)}
		\end{itemize}
		\item \flushleft Verlust des Überblicks bei 20 Kopien der selben Datei mit anderem Namen
	\end{itemize}
	\pause{}
	\vfill
	\textbf{\alert{Lösung:}}
	\begin{itemize}
		\item immer selbe Datei bearbeiten
		\item Zwischenstand irgendwo speichern, mit Info, was geändert wurde (zum rückgängig machen)
	\end{itemize}
\end{frame}

\begin{frame}{Wie das aussehen könnte...}
	\centering \includegraphics[width=\textwidth, keepaspectratio]{assets/04_git/git-diff}

	\textcolor{green}{grün:} hinzugefügte Zeilen
	% TODO: evtl. pädagogisch wertvolle Anmerkungen zum Bildinhalt hier
\end{frame}

\begin{frame}{weitere Vorteile}
	% \includegraphics[width=.8\textwidth, keepaspectratio]{assets/04_git/branch}
	\begin{itemize}
		\item man sieht, wer Fehler ins Programm eingeschleust hat
		\item zurück springen auf jeden beliebigen vorherigen Stand \\
			$\rightarrow$ jede Änderung \emph{einzeln} rückgängig machbar
		\item auch online speicherbar
	\end{itemize}
\end{frame}

\begin{frame}{Programme zur Versionsverwaltung}
	Bekannte Programme:
	\begin{itemize}
		\item SVN
		\item Mercurial
		\item Git \uncover<2->{- \textbf{zurzeit das Beliebteste}}
	\end{itemize}
\end{frame}

\begin{frame}
	\centering \includegraphics[width=.6\textwidth, keepaspectratio]{assets/04_git/git-logo} \\[.5cm]
\end{frame}

% kleine praktische Einführung in Git, also packt die Laptops aus \o/


% kann übersprungen werden, wenn schon alle git haben
\subsection{Installation}

\begin{frame}
	\centering \Huge \textbf{Installation}
\end{frame}

\begin{frame}{Installation - Linux}
	\centering \textbf{Ubuntu/Debian} \\
	\texttt{\$ sudo apt-get install git}

	\vfill

	\textbf{Arch Linux} \\
	\texttt{\$ sudo pacman -Sy git}

	\vfill

	\textbf{Fedora} \\
	\texttt{\$ dnf -y install git}
\end{frame}

\begin{frame}{Installation - macOS}
	\centering
	\textbf{Über Homebrew} (empfohlen)\\
	\texttt{\$ brew install git}

	\vfill

	\textbf{Download von der Webseite} \\
	\url{https://git-scm.com}
\end{frame}

\begin{frame}{Installation - Windows}
	\centering
	\textbf{Installation des Linux Subsystems} \\
	Danach, wie bei der gewählten Linuxdistibution
\end{frame}


\begin{frame}{Clients}
	Es gibt eine Reihe von grafischen Clients, die auf dem Kommandozeilenprogramm aufsetzen.
	Empfehlenswert sind:
	\begin{itemize}
		\item GitKraken
		\item SourceTree (viele Konfigurationsmöglichkeiten)
		\item GitHub Desktop
		\item IDE 
	\end{itemize}

	\vfill
	\alert{Aaaaber:} Lieber Kommandozeilen-Client nutzen, um die Funktionsweise zu lernen.
\end{frame}


\subsection{Einführung}
\begin{frame}
	\centering \Huge \textbf{Hands on!} \\[.7cm]
	\normalsize Zeit für eine praktische Einführung!
\end{frame}

\begin{frame}{Vorbereitung}
	\begin{itemize}
		\item [$\Box$] \git installiert?
		\item [$\Box$] neuen (leeren) Ordner anlegen
		\item [$\Box$] Linux Subsystem (Windows) oder Terminal öffnen \\
			per \alert{\texttt{cd pfad/zum/ordner}} in den neuen Ordner wechseln
	\end{itemize}
\end{frame}
\begin{frame}{Navigation in einem Terminal}
	\centering \textbf{noch nie eine Shell benutzt?}
	\pause
	\begin{itemize}
		\item \alert{\texttt{cd mein\_ordner}} wechsle in \texttt{mein\_ordner}
		\item \alert{\texttt{mkdir mein\_ordner}} erstelle \texttt{mein\_ordner}
		\item \alert{\texttt{ls}} liste mir alle Datein/Ordner im aktuellen Verzeichnis
		\item \alert{\texttt{touch meine\_datei}} erstelle \texttt{meine\_datei}
		\item \alert{\texttt{cat meine\_datei}} zeige den Inhalt von \texttt{meine\_datei} an
		\item \alert{\texttt{rm meine\_datei}} lösche \texttt{meine\_datei}
	\end{itemize}
\end{frame}
\begin{frame}[fragile]{Initiales Setup}
	\centering \textbf{Noch nie \git benutzt?}
	\vfill
	\pause{}
	
	\centering \large \textbf{\alert{\texttt{git config --global user.name "[name]"}}} \\[.2cm]
	\normalsize Setzt Namen, der unter dem Commit stehen wird.
	
	\vfill
	\pause{}
	
	\large \textbf{\alert{\texttt{git config --global user.email "[email]"}}} \\[.2cm]
	\normalsize Mailadresse des Users (\textbf{wichtig:} sollte GitHub-Adresse sein!)
	
	\vfill
	\pause{}
	
	\begin{lstlisting}[basicstyle=\scriptsize\ttfamily]
[felix@Samaritan] $ git config --global user.name "Max Mustermann"
[felix@Samaritan] $ git config --global user.email "mustermann@ifsr.de"
\end{lstlisting}
\end{frame}

% TODO: Hier Grundprinzip des "lagers" noch mal erklären?

\begin{frame}[fragile]{Und am Anfang war... }
	\centering \Large \textbf{\texttt{\alert{git init}}}\\[.2cm]
	%\vfill
	\normalsize Erstellt ein neues Repository \emph{(Lager)} im aktuellen Ordner.
	\pause{}

	\vfill
	\begin{lstlisting}
[felix@Samaritan] $ git init
Initialized empty Git repository in /Users/felix/test/.git/
\end{lstlisting}

\end{frame}

\begin{frame}{Und was ist jetzt passiert?}
	\begin{itemize}
		\item \texttt{.git}-Ordner wurde angelegt
		\item in ihm geschieht \glqq{}Magie\grqq{} hinter den Kulissen
		\item dient als Ablage für gespeicherte Änderungen an Dateien
		\item Einstellungen für Repository liegen hier
	\end{itemize}

	In diesem Ordner müsst ihr für gewöhnlich nichts ändern.
\end{frame}

% Theorie-Exkurs
\begin{frame}
	\centering \Large \textbf{Ein kleiner Blick unter die Haube...}
\end{frame}

\begin{frame}{\git - Workflow}
	Intern organisiert \git die überwachten Dateien in 3 Bereichen:
	
	\begin{description}[Working Directory]
		\item [Working Directory] Da liegen alle bearbeiteten Dateien aus eurem Projekt-Ordner drin.
		\item [Staging Area] Wollt ihr euren Zwischenstand speichern, könnt ihr einzelne Änderungen (einzelne Dateien oder sogar Zeilen) dafür auswählen. Diese landen dann hier.
		\item [Repository] Nachdem die ausgewählten Änderungen aus der Staging Area committed wurden, liegen sie als Gesamtpaket (Commit) hier.
	\end{description}
\end{frame}

\begin{frame}<1-3>[label=workflow]{\git - Workflow}
	\centering
	%TODO: Make my colors pretty!
	\begin{tikzpicture}
	\tikzstyle{block} = [font=\bfseries, text = white, rounded corners=2pt, minimum height = 2.5em, text width=9em, text centered]
		\node[block, fill=red] (dir) at (3.5,6) {Working Direktory};
		\uncover<2->{
			\node (add) at (0,5) {\texttt{git add}};
			\node[block, fill=orange] (area) at (3.5,4) {Staging Area};
			\path [draw] (dir) -| (add);
			\path [draw, -latex'] (add) |- (area.+175);
			\node (reset) at (7,5) {\texttt{git reset}};
			\path [draw] (area) -| (reset);
			\path [draw, -latex'] (reset) |- (dir);
		}
		\uncover<3->{
			\node (commit) at (0,3) {\texttt{git commit}};
			\node[block, fill=blue] (repo) at (3.5,2) {Repository};
			\path[draw] (area.-175) -| (commit);
			\path[draw, -latex'] (commit) |- (repo.+175);
		}
		\only<4->{
			\node[block, fill=green](remote) at (3.5,0) {Remote Repository};
		\uncover<5->{
			\node (push) at (0,1) {\texttt{git push}};
			\path[draw] (repo.-175) -| (push);
			\path[draw, -latex'] (push) |- (remote);
			\node (pull) at (7,1) {\texttt{git pull}};
			\path[draw] (remote) -| (pull);
			\path[draw, -latex'] (pull) |- (repo);
		}}
	\end{tikzpicture}
\end{frame}

\begin{frame}
	\centering \Large \textbf{Zurück zur Praxis!}
\end{frame}

\begin{frame}[fragile]{Her mit dem Inhalt!}
	\centering
	\footnotesize \emph{(Normalerweise kommt jetzt der Teil, an dem ihr programmiert.)} \\
	\normalsize Erstellt eine Textdatei und schreibt ein paar Zeilen.

%	\vfill
%	\begin{lstlisting}
%[felix@Samaritan] $ echo "Hallo Welt! :)" > hallowelt.txt
%\end{lstlisting}
\end{frame}

\begin{frame}{Änderungen speichern}
	\centering Genug Inhalt zusammengekommen? \\
	Zeit für den \emph{Commit}.

	\vfill
	\pause{}

	\textbf{Den was...?}
\end{frame}

\begin{frame}{Commitment leicht gemacht}
	Ein \alert{Commit} ist der Vorgang, bei dem die geänderten Dateien aus dem Working Tree gespeichert werden.

	Das spielt sich nach dem immer gleichen Schema ab:
	\begin{enumerate}
		\item Dateien bearbeiten
		\item zu speichernde Änderungen auswählen (\alert{add}n oder stagen)
		\item Änderungen \emph{\alert{commit}ten}
	\end{enumerate}
\end{frame}

\begin{frame}[fragile]{Änderungen speichern (1)}
	\centering Mit \alert{\texttt{git status}} wird Status des Repositories angezeigt:\\[.7cm]

	\begin{lstlisting}
[felix@Samaritan] $ git status
On branch main

Initial commit

§Untracked files:§
  (use "git add <file>..." to include in what will be committed)

    hallowelt.txt

nothing added to commit but untracked files present (use "°git add°" to track)
\end{lstlisting}
\end{frame}

\begin{frame}[fragile]{Änderungen speichern (2)}
	\centering Jetzt müssen alle geänderten Dateien \emph{gestaged} werden:

	\Large \textbf{\texttt{\alert{git add [Datei(en)]}}}\\[.7cm]

	\begin{lstlisting}
[felix@Samaritan] $ git add hallowelt.txt
\end{lstlisting}
	\normalsize
	\pause{}

	\begin{center} Immer überlegen, \emph{welche} Dateien geaddet werden sollen.
	Passwörter und Access-Token haben in git nichts zu suchen! \end{center}
\end{frame}

\begin{frame}{Wo stehen wir jetzt?}
	\centering Probiert doch noch mal \texttt{git status}...
\end{frame}

\begin{frame}[fragile]{Wo stehen wir jetzt?}
	\begin{lstlisting}
[felix@Samaritan] $ git status
On branch main

Initial commit

Changes to be committed:
  (use "git rm --cached <file>..." to unstage)

    ^new file:   hallowelt.txt^
\end{lstlisting}
\end{frame}

\begin{frame}[fragile]{Zeit, den Sack zu zu machen!}
	\centering Bleibt nur noch der eigentliche \emph{Commit} übrig!

	\Large \textbf{\texttt{\alert{git commit -m "[Nachricht]"}}}\\[.7cm]

	\begin{lstlisting}
[felix@Samaritan] $ git commit -m "Add hallowelt.txt"

[main (root-commit) 6386d2f] Add hallowelt.txt
 1 file changed, 1 insertion(+)
 create mode 100644 hallowelt.txt
\end{lstlisting}
	\normalsize
	
	\begin{center} Glückwunsch! Euer erster Commit! :) \end{center}
\end{frame}

\begin{frame}
	\begin{description}[Tipp:]
		\item [\alert{Tipp:}] Verwendet \emph{aussagekräftige} Commit-Nachrichten, aus denen man sofort herauslesen kann, welche Änderungen ihr vorgenommen habt!
	\end{description}
	\vfill
	\textbf{Negativ-Beispiel:}\\[.2cm]
	\centering \includegraphics[width=.85\textwidth, keepaspectratio]{assets/04_git/bad-commit-msg}
\end{frame}

\begin{frame}
	\centering\Large\textbf{Nett. Aber alles immer noch lokal. Wie funktioniert jetzt die Zusammenarbeit?}\\[.7cm]
	\pause
	\alert{Mit Remote Repositories!}
\end{frame}

\againframe<3->{workflow}

\begin{frame}{Ab in die Cloud! - Befehle}
	\centering \Large \textbf{\alert{\texttt{git push [Remote] [Branch]}}} \\[.2cm]
	\normalsize lädt eure Commits hoch (man nennt es auch \emph{pushen})
	\vfill

	\Large \textbf{\alert{\texttt{git pull [Remote] [Branch]}}} \\[.2cm]
	\normalsize lädt fremde Änderungen runter (man nennt es auch \emph{pullen})
	\vfill
	\pause{}
	Aber dazu brauchen wir erstmal ein Remote Repository...
\end{frame}

\begin{frame}
	\centering \includegraphics[width=.95\textwidth, keepaspectratio]{assets/04_git/github/01}
\end{frame}

\begin{frame}
	\centering \includegraphics[width=.95\textwidth, keepaspectratio]{assets/04_git/github/02}
\end{frame}

\begin{frame}
	\centering \includegraphics[width=.95\textwidth, keepaspectratio]{assets/04_git/github/03}
\end{frame}

\begin{frame}[fragile]{Ab in die Cloud! (2)}
	\begin{lstlisting}
[felix@Samaritan] $ git remote add origin https://github.com/Feliix42/cooles-test-repo.git
[felix@Samaritan] $ git push -u origin main
Username for 'https://github.com': Feliix42
Password for 'https://Feliix42@github.com':
Counting objects: 3, done.
Writing objects: 100% (3/3), 869 bytes | 0 bytes/s, done.
Total 3 (delta 0), reused 0 (delta 0)
To https://github.com/Feliix42/cooles-test-repo.git
 * [new branch]      main -> main
Branch main set up to track remote branch main from origin.
\end{lstlisting}
\end{frame}

\begin{frame}
	\centering \includegraphics[width=.95\textwidth, keepaspectratio]{assets/04_git/github/04}

	Es funktioniert!
\end{frame}

\begin{frame}<1-2>[label=collaboration]{Zum Abschluss: Kollaboration!}
	\begin{itemize}
		\item<1-> findet euch zu zweit zusammen
		\item<2-> Partner 1 fügt Partner 2 zu seinem Repository hinzu
		\item<3-> Partner 2 klont das Repository von Partner 1
		\item<4-> Bearbeitet beide \texttt{hallowelt.txt} \\
			(fügt z.B. den Satz \glqq{}Ich heiße [...]\grqq{} hinzu und löscht den Rest.)
		\item<5-> Einer von euch committed und pusht seine Änderungen \\
			(\texttt{git add [...]}, \texttt{git commit -m "[...]"}, \texttt{git push origin main})
		\item<6-> Dann macht der andere Partner das gleiche
	\end{itemize}
\end{frame}

\begin{frame}
	\centering \includegraphics[width=.95\textwidth, keepaspectratio]{assets/04_git/github/05}
\end{frame}

\againframe<2-3>{collaboration}

\begin{frame}
	\centering \includegraphics[width=.95\textwidth, keepaspectratio]{assets/04_git/github/06}
\end{frame}

\begin{frame}[fragile]
	\begin{lstlisting}
[harold@Machine] $ git clone https://github.com/Feliix42/cooles-test-repo.git
Cloning into 'cooles-test-repo'...
remote: Counting objects: 3, done.
remote: Total 3 (delta 0), reused 3 (delta 0), pack-reused 0
Unpacking objects: 100% (3/3), done.
Checking connectivity... done.
\end{lstlisting}
\end{frame}

\againframe<3-6>{collaboration}

\begin{frame}[fragile]
	\begin{lstlisting}
[felix@Samaritan] $ git commit -m "Ich bin Felix"

[main 50126f1] Ich bin Felix
 1 file changed, 1 insertion(+), 1 deletion(-)

[felix@Samaritan] $ git push origin main
To https://github.com/Feliix42/cooles-test-repo.git
 °! [rejected]        main -> main (fetch first)°
error: failed to push some refs to 'https://github.com/Feliix42/cooles-test-repo.git'
hint: Updates were rejected because the remote contains work
hint: that you do not have locally. This is usually caused
hint: by another repository pushing to the same ref. You may
hint: want to first integrate the remote changes (e.g.,
hint: 'git pull ...') before pushing again. See the 'Note
hint: about fast-forwards' in 'git push --help' for details.
\end{lstlisting}
\begin{center}\textbf{Huch.}\end{center}
\end{frame}

\begin{frame}
	\centering Euer \texttt{push} wurde zurückgewiesen, weil es neuere Änderungen gibt, die ihr noch nicht auf eurem PC habt.
	\vfill

	\textbf{Lösung:}
	
	\texttt{git pull origin main}
\end{frame}

\begin{frame}[fragile]
	\begin{lstlisting}
[felix@Samaritan] $ git pull origin main
remote: Counting objects: 3, done.
remote: Total 3 (delta 0), reused 3 (delta 0), pack-reused 0
Unpacking objects: 100% (3/3), done.
From https://github.com/Feliix42/cooles-test-repo
   6386d2f..90188bf  main     -> origin/main
Auto-merging hallowelt.txt
°CONFLICT° (content): °Merge conflict° in hallowelt.txt
Automatic merge failed; fix conflicts and then commit the result
\end{lstlisting}

\pause{}
\begin{center} \textbf{} Glückwunsch! Euer erster \emph{Merge-Conflict!} \end{center}
\end{frame}

\begin{frame}
	\centering \includegraphics{assets/04_git/git-conflict}
\end{frame}

\begin{frame}{Okay... Und jetzt?}
	\centering \includegraphics[width=.9\textwidth, keepaspectratio]{assets/04_git/merge-conflict}
	\vfill

	Überschneiden sich die Änderungen von zwei Commits, die \git nicht automatisch beheben kann, entsteht ein \textbf{Merge Conflict}, der von Hand behoben werden muss.
\end{frame}

\begin{frame}{Wie?}
	\centering Öffnet \texttt{hallowelt.txt} im Texteditor. \\
	Das sollte etwa so aussehen:


	\pause{}
	\includegraphics[width=.9\textwidth, keepaspectratio, trim=0 400 300 0, clip]{assets/04_git/resolve_this}
\end{frame}

\begin{frame}<1-3>[label=resolve]{Was sehe ich da?}
	\begin{itemize}
		\item oberhalb des \texttt{=======} stehen eure Änderungen (markiert mit \emph{HEAD})
		\item darunter die Änderung des Konflikte verursachenden Commits (markiert mit der \emph{Commit-ID})
	\end{itemize}

	\pause{}
	\begin{itemize}
		\item kompletter Abschnitt zwischen \texttt{<<<<} und \texttt{>>>>} muss jetzt von Hand gemerged werden
		\item löscht die Änderungen, die weg sollen und fasst alles so zusammen, dass es euch passt
		\item<3-> seht euch mit \texttt{git status} den Zwischenstand an
		\item<4-> addet die \glqq{}reparierte\grqq{} Datei
		\item<4-> commited das Ergebnis
	\end{itemize}
\end{frame}

\begin{frame}[fragile]
	\begin{lstlisting}
[felix@Samaritan] $ git status
On branch main
Your branch and 'origin/main' have diverged,
and have 1 and 1 different commits each, respectively.
  (use "git pull" to merge the remote branch into yours)
You have unmerged paths.
  (fix conflicts and run "git commit")
  (use "git merge --abort" to abort the merge)

Unmerged paths:
  (use "git add <file>..." to mark resolution)

    §both modified:   hallowelt.txt§

no changes added to commit (use "git add" and/or "git commit -a")
\end{lstlisting}
\end{frame}

\againframe<3-4>{resolve}

\begin{frame}{Merge Konflikte sind kein Beinbruch!}
	\begin{columns}[onlytextwidth]
		\begin{column}{0.6\textwidth}
			Generell gilt: Ruhe bewahren!
			\begin{itemize}
				\item Merge Konflikte immer mit Sorgfalt beheben, besonders wenn große Code-Segmente betroffen sind!
				\item Ein Konflikt kann sich über mehrere Blöcke erstrecken
				\item schaut euch betroffene Dateien genau an!
			\end{itemize}
		\end{column}
		\begin{column}{0.4\textwidth}
			\flushright \includegraphics[width=0.85\textwidth, keepaspectratio]{assets/04_git/git}
		\end{column}
	\end{columns}
\end{frame}

\begin{frame}
	\centering \Large \alert{Glückwunsch!}
	\vfill

	\normalsize Ihr könnt jetzt:
	\begin{itemize}
		\item Euer Projekt mit \git verwalten
		\item euer Repository mit GitHub über mehrere PCs und User synchronisieren
		\item Merge Konflikte beheben
		\item fremde Projekte klonen und mitmachen!
	\end{itemize}
	Ihr seid für eure ersten Schritte mit \git gewappnet!
\end{frame}

\begin{frame}
	\centering \Large \alert{Doch mit git ist noch viel mehr möglich!}
	\normalsize
	\vfill
	\begin{columns}[onlytextwidth]
		\hspace*{-.05\textwidth}
		\begin{column}{0.6\textwidth}
			\centering \includegraphics[width=0.85\textwidth, keepaspectratio]{assets/04_git/yoda}
		\end{column}
		\begin{column}{0.5\textwidth}
			Werft zum Beispiel mal ein Blick auf folgende Dinge:
			\begin{itemize}
				\item die \texttt{.gitignore}-Datei
				\item Branches
				\item \texttt{git log} \\auch mit den Zusätzen
				\begin{itemize}
					\item[] \texttt{--oneline}
					\item[] \texttt{--graph}
				\end{itemize}
				\item \texttt{git diff}
				\item \texttt{git pull --rebase}
				\item \texttt{git commit --amend}
				\item \texttt{git fetch} vs. \texttt{git pull}
			\end{itemize}
		\end{column}
	\end{columns}
\end{frame}

\begin{frame}{Hilfe, so viele Befehle! Das merk ich mir nie!}
	Musst du auch nicht, es gibt ja Cheat Sheets:
	\begin{itemize}
		%\item \href{https://services.github.com/on-demand/downloads/github-git-cheat-sheet.pdf}{GitHub git Cheat Sheet}
		\item \href{https://education.github.com/git-cheat-sheet-education.pdf}{GitHub git Cheat Sheet Education}
		\item \href{https://www.git-tower.com/blog/git-cheat-sheet/}{Cheat Sheet von Tower}
		\item und viele mehr - einfach mal googlen
	\end{itemize}
	Außerdem gibt es zahlreiche Guides und Einführungen im Netz:
	\begin{itemize}
		\item \href{https://rogerdudler.github.io/git-guide/index.de.html}{git - Der einfache Einstieg}
		\item \href{http://juristr.com/blog/2013/04/git-explained/}{Git Explained: For Beginners}
		\item und viele mehr, schaut einfach, was euch anspricht!
	\end{itemize}
	\begin{description}[Tipp:]
		\item[\alert{Tipp:}] Unbedingt über Good Practices informieren!
	\end{description}
\end{frame}


% git commit ...
%   -> Änderungen zu einem Paket zusammenfassen, dass dann als gesamtes Rückgängig gemacht werden kann (bzw. Stand, zu dem man springen kann)
% git push ...
%   -> Wenn man online ein Repo hat, kann man hier alles hoch laden, alle Änderungen werden einzeln übertragen, sodass alle die Änderungen einzeln sehen können
%
% -> in dem zusammenhang auch gitHub zeigen (noch mal für das education pack werben!)
% super Seite, um projekt online zu sichern (gibt aber auch private gehostete Alternativen)
% kann man sich trauen, vom fusionforge abzuraten? -> ansonsten gekonnt auslassen
%
% git pull
%   -> herunterladen der Änderungen, die andere gemacht haben
%
%
% git checkout -b new_branch
% arbeit
% Commit
%
% git merge
%   -> was, wenn sich Änderungen mehrerer Menschen überschneiden?
%
%
% generell:
% git clone
%   -> Herunterladen der gesamten history eines Projektes (z.B. um mit zu machen)
%   -> Zeigen!



% Good Practices:
% - Master-Branch immer sauber halten (am besten auf anderem Branch arbeiten)
% - gitignore Dateien benutzen
% - https://education.github.com/git-cheat-sheet-education.pdf

\section{Ausblick}

%\begin{frame}{Linux Install Party}
%	\begin{columns}[onlytextwidth]
%		\begin{column}{0.5\textwidth}
%			\includegraphics[width=0.85\textwidth, keepaspectratio]{assets/04_git/hacken101_linux_install_party}
%		\end{column}
%		\begin{column}{0.5\textwidth}
%			\large Lust, ein Linux-basiertes Betriebssystem zu installieren? \\[.5cm]
%			\normalsize Bei der \alert{Linux Install Party} werdet ihr gemeinsam mit euren Kommilitonen eine Linux-Distribution eurer Wahl installieren. \\[.25cm]
%			Dabei steht man euch natürlich mit Rat und Tat zur Seite.
%		\end{column}
%	\end{columns}
%\end{frame}

%\begin{frame}{NEU: Hacken 101}
%	\begin{columns}[onlytextwidth]
%		\begin{column}{0.5\textwidth}
%			\includegraphics[width=0.85\textwidth, keepaspectratio]{assets/04_git/hacken101_unix_vim}
%		\end{column}
%		\begin{column}{0.5\textwidth}
%			\large Die Veranstaltungsreihe \alert{Hacken 101} stellt Themen abseits der normalen Lehrveranstaltungen vor. \\[.5cm]
%			\normalsize Unter anderem zu:
%			\begin{itemize}
%				\item Unix \& Vim
%				\item SSH
%				\item PHP
%				\item Git
%			\end{itemize}
%		\end{column}
%	\end{columns}
%\end{frame}

%\begin{frame}{Wahlen}
%	\centering \Large Der FSR soll \textbf{deine} Interessen vertreten!\\Deine Stimme zählt.\\
%	\Huge \alert{\textbf{Also geh wählen!}}
%	
%	\normalsize
%	\flushleft
%	
%	\textbf{Jeweils von 9 bis 17 Uhr:}\\
%	\begin{itemize}
%	\item Dienstag, 27.11. - HSZ
%	\item Mittwoch, 28.11. - APB (direkt im Foyer)
%	\item Donnerstag, 29.11. - APB (direkt im Foyer)
%	\end{itemize}
%	
%	\textbf{Nicht in Dresden?}\\Bis 22.11. persönlich Unterlagen für Briefwahl besorgen!
%	
%	Weitere Infos findet ihr unter \url{https://www.ifsr.de/wahlen}.
%\end{frame}

%\begin{frame}{Nächster Spieleabend}
%		\medskip
%		\begin{columns}[onlytextwidth]
%			\begin{column}{0.5\textwidth}
%				\begin{itemize}
%					\item \textbf{Wann?} 30. Januar
%					\item \textbf{Wo?} Erdgeschoss des APB
%					\item \textbf{Wie spät?} 18:30 Uhr
%				\end{itemize} \ \\[.25cm]
%
%				Kommt vorbei, vergesst den Uni-Alltag und habt Spaß! :) \\ \ \\
%				Für Knabbereien und Getränke ist gesorgt!
%			\end{column}
%			\begin{column}{0.5\textwidth}
%				\centering \includegraphics[width=0.9\textwidth, keepaspectratio]{assets/04_git/spieleabend}
%		\end{column}
%		\end{columns}
%\end{frame}

\begin{frame}{Nächstes Treffen}
	\centering \Huge \alert{\textbf{Nächster Termin:}} \\
	\Large \neuerTermin \\[.7cm]
	\normalsize\flushleft
	Freut euch auf:
	\begin{itemize}
		\item Wie geht es weiter? - Ausblick auf kommende Semester
		\item Tipps zum Stundenplanbau
		\item Wie gehe ich mit dem Prüfungsamt um?
		\item Und\dots ein paar letzte Tipps bevor ihr in die Welt der höheren Semester entlassen werdet :)
	\end{itemize}
\end{frame}

\end{document}
